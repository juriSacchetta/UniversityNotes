\chapter{Practical exercises}
    \section{Exercise 1}
        \subsection{Indistinguishable states}
            While speaking of clean automaton \ref{sect:cleanautom}, a \emph{practical} translation of the definition \ref{def:undiststates}, chapter is: two states are 
            undistinguishable if the outgoing arcs with the same label brings to the same state (and their outgoing arcs have the same lable), or, in other words, 
            equal lable $\Rightarrow$ same state.
        \subsection{Linear Language Equations}\label{ese:linlangeq}
            Referring to section \ref{sect:linlangeq}
            In order to pass from grammar rules to linear language equations, substitute the $\vert$ symbol with $\bigcup$, and each nonterminal with a sublanguage.
        \subsection{Local automaton / language}\label{sect:practlocautom}
            Referring to definition \ref{def:loclang}, when we are asked to define whether a language is local and we have the minimal automaton available, we can check 
            whether the automaton is local. In order to do so, we can say that the automaton is local when does not exist two arcs with different labels going into 
            different states (needs to be minimal otherwise the arcs could end up in two different states, which are the same in reality).

            In case the automaton is not available, we have to find $Dig$, $Fin$, $Ini$ and check whether all their components can be recognized.
        \subsection{Ambiguity of automaton}
            For definition of \index{Ambiguity}ambiguity refer to the section \ref{sec:ambig}.
            An automaton is ambiguous if enables two or more accepting paths for the same string. Note that also epsilon path: in fact, if you find an epsilon loop
            it is ofr sure ambiguous.
    \section{Exercise 2.1}
        \subsection{Regular languages and context free}
            Remember: a context free language example can be $a^nb^n$. A regular language is context free, but a context free language is not always regular: for example
            $a^nb^n$ is context free but not regular. Remember that context free is \textbf{not} closed with respect to $\cap$. An intersection of two context free 
            languages \textbf{may} still \textbf{be} context free, but it is not guaranteed.
            %//TODO: aggiungere teoria di questa parte
        \subsection{Deterministic vs. non deterministic automaton from grammar}
            Given a grammar, how to check whether the corresponding automaton is deterministic or not? In a deterministic automaton, given an input I need to know for 
            sure in which state I will end. Epsilon moves makes the automaton nondeterministic, and a copy rule corresponds exactly to an epsilon move (spontaneous move):
            So, if you have an alternative such as:
            $$X\rightarrow aT \vert \varepsilon$$
            $$T\rightarrow aT\vert V$$
            $$V\rightarrow bV\vert cV\vert\varepsilon$$
            The automaton is nondeterministic: in $T$ is like if you don't know whether to spontaneously go in $V$ or stay in $T$ to wait for an $a$.
        \subsection{Is a language regular?}
            If it is writable through a regex, it is. If it is the intersection of two regular languages, it is ($REG$ is closed with respect to the intersection).
        \subsection{Automata, languages and grammars}
            \begin{center}
                \begin{tabular}{cccc}
                    \toprule
                    \textbf{\tableline{Chomsky\\hierarchy}} & \textbf{Grammars} & \textbf{Languages} & \textbf{\tableline{Minimal\\Automaton}}\\
                    \midrule
                    Type $0$ & Generals & \tableline{Recursively\\numerable} & Turing Machine\\
                    Type $1$ & Context dependent & Context dependent & \tableline{Linear Bounded\\Automaton\\(LBA)}\\
                    Type $2$ & Non contextual & Non contextual & NPDA\\
                    Type $3$ & Regular & Regular & FSA\\
                    \bottomrule
                \end{tabular}
            \end{center}
        \subsection{Closure Properties of Languages}
            \begin{center}
                \begin{tabular}{cccccccc}
                    \toprule
                    \multirow{3}{*}{\textbf{\tableline{Language\\Family}}} & \multirow{3}{*}{\textbf{Formalism}} & $\cup$ & $\cap$ & $c$ & $/$ & $*$ & $.$ \\
                    \cmidrule{3-8}
                    && \textbf{Union} & \textbf{Inters.} & \textbf{Complem.} & \textbf{Diff.} & \textbf{\tableline{Klenee\\star}} & \textbf{Concat.}\\
                    \midrule
                    Star free & MFO & V & V & V & V & X & V \\
                    \midrule
                    Regular & \tableline{FSA,NFA\\Regex,\\Regular\\grammars,\\MSO} & V & V & V & V & V & V\\
                    \midrule
                    \tableline{Deterministic\\non\\contextual} & DPDA & X & X & V & X & X & X\\
                    \midrule
                    \tableline{Non\\contextual} & \tableline{NPDA,\\Non\\contextual\\grammars}& V & X & X & X & V & V\\
                    \midrule
                    \tableline{Context\\dependent} & \tableline{context\\depending\\grammar\\(Linear\\bounded\\automaton)} & V & V & V & V & V & V\\
                    \midrule
                    Recursive & (Decider) & V & V & V & V & V & V\\
                    \midrule
                    \tableline{Recursively\\Enumerable} & \tableline{TM,NTM,\\General\\Grammars} & V & V & X & X & V & V\\
                    \bottomrule 
                \end{tabular}
            \end{center}
