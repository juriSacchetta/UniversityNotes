\chapter{Practical exercises}
    \section{Exercise 1}
        \subsection{Indistinguishable states}
            While speaking of clean automaton \ref{sect:cleanautom}, a \emph{practical} translation of the definition \ref{def:undiststates}, chapter is: two states are 
            undistinguishable if the outgoing arcs with the same label brings to the same state (and their outgoing arcs have the same lable), or, in other words, 
            equal lable $\Rightarrow$ same state.
        \subsection{Linear Language Equations}\label{ese:linlangeq}
            Referring to section \ref{sect:linlangeq}
            In order to pass from grammar rules to linear language equations, substitute the $\vert$ symbol with $\bigcup$, and each nonterminal with a sublanguage.
        \subsection{Local automaton / language}\label{sect:practlocautom}
            Referring to definition \ref{def:loclang}, when we are asked to define whether a language is local and we have the minimal automaton available, we can check 
            whether the automaton is local. In order to do so, we can say that the automaton is local when does not exist two arcs with different labels going into 
            different states (needs to be minimal otherwise the arcs could end up in two different states, which are the same in reality).

            In case the automaton is not available, we have to find $Dig$, $Fin$, $Ini$ and check whether all their components can be recognized.