\chapter{Regular Expressions and Regular Languages}
    \section{Introduction}
        Regual languages are the simplest family of formal language.
    \section{RL and RE}
        \subsection{Introduction and Definitions}
            A Regular Language is a \textbf{language over an alphabet that can be expressed by applying} \emph{a finite number of times} this three operations:
            \begin{enumerate}
                \item Union $\bigcup, \vee$
                \item Concatenation $\cdot$
                \item Star (Kleene Star)
            \end{enumerate}
            Rule of precedence: star, concatenation, union.
            It is permitted to use $\varepsilon = \emptyset^* $
            
            \begin{definition}[Regular Language]
                A language is regular  if it is the meaning (hence if it is generated) by a regular expression.
            \end{definition}
            \begin{definition}[Family of Regular Languages (REG)]
                Collection of regular languages.
            \end{definition}
            \begin{definition}[Family of Finite Languages (FIN)]
                Collection of all languages of finite cardinality.
            \end{definition}
            \textbf{Caution}: REG and FIN are sets of sets of set of strings, not set of strings themeselves.

            Notice that \emph{every finite language is regular} because it is the (finite)
            union of finitely many strings, and each string is in turn the concatenation
            of finitely many letters or in general of alphabetic symbols

            \begin{equation*}
                \left(x_1\bigcup x_2\bigcup\ldots\bigcup x_k\right)=\left({a_1}_1{a_1}_2\ldots {a_1}_n\bigcup\ldots\bigcup{a_k}_1{a_k}_2\ldots{a_k}_n\right)
            \end{equation*}

            Remember that $FIN\subset REG$
        \subsection{Subexpression of a regexp}
            \begin{enumerate}
                \item Consider a fully parenthesized regex
                \begin{equation}
                    e=\left(a\bigcup\left(bb\right)\right)^*\left(c^+\bigcup\left(a\bigcup\left(bb\right)\right)\right)
                \end{equation}
                \item Number every alphabetic symbol that occurs in the regexp
                \begin{equation}
                    e_N=\left(a_1\bigcup\left(b_2b_3\right)\right)^*\left(c_4^+\bigcup\left(a_5\bigcup\left(b_6b_7\right)\right)\right)
                \end{equation}
                \item isolate the subexpressions and put them into evidence
            \end{enumerate}



        \subsection{Other Operators}
            Also set operators can be used: Intersection and Complement are used when it's more concise to express a language to a set of languages (the former) or 
            the inverse of a complex rule (the latter: a clear example is "a language with no 2 consecutive 'a' in it").
        
        To define a Regular Language formally a \emph{Regular Expression} is used, defined as a string of either elements of the alphabet or operators that symbolize 
        the three operation for regular languages.
        The Derivation Relation binds the RE with its productions: in fact, an expression is said to be \emph{derived} from another one if it's totally or partly a 
        choice of another one. A choice is an alternative between two options that the RE gives you (a union operator poses a choice). A Language is defined as the 
        set of all possible productions that can be derived from a RE.

    \section{Ambiguity}
        Easy enough: a RE is ambiguous if more than one identical productions exists. This obviously leads to problems in the parsing procedure, due to the difficulty 
        to "trace back" the choices that has been made to get to that point. 

    \section{Closure Property}
        Pretty straightforward, is the same "closure" concept from classic algebra; a family of languages is closed under an operator if applying n times that 
        operator to a language of that family obtains only languages of that family. You can't "exit that family" with that operator.
        
        The REG(ular) family of languages is closed under
        \begin{itemize}
            \item concatenation
            \item union
            \item star
        \end{itemize}
        Therefore is closed also under derived operators from these (cross).
