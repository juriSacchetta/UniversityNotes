% A LaTeX template for MSc Thesis submissions to 
% Politecnico di Milano (PoliMi) - School of Industrial and Information Engineering
%
% S. Bonetti, A. Gruttadauria, G. Mescolini, A. Zingaro
% e-mail: template-tesi-ingind@polimi.it
%
% Last Revision: October 2021
%
% Copyright 2021 Politecnico di Milano, Italy. NC-BY

\documentclass{Configuration_Files/PoliMi3i_thesis}

%------------------------------------------------------------------------------
%	REQUIRED PACKAGES AND  CONFIGURATIONS
%------------------------------------------------------------------------------

% CONFIGURATIONS
\usepackage{parskip} % For paragraph layout
\usepackage{setspace} % For using single or double spacing
\usepackage{emptypage} % To insert empty pages
\usepackage{multicol} % To write in multiple columns (executive summary)
\setlength\columnsep{15pt} % Column separation in executive summary
\setlength\parindent{0pt} % Indentation
\raggedbottom  

% PACKAGES FOR TITLES
\usepackage{titlesec}
% \titlespacing{\section}{left spacing}{before spacing}{after spacing}
\titlespacing{\section}{0pt}{3.3ex}{2ex}
\titlespacing{\subsection}{0pt}{3.3ex}{1.65ex}
\titlespacing{\subsubsection}{0pt}{3.3ex}{1ex}
\usepackage{color}

% PACKAGES FOR LANGUAGE AND FONT
\usepackage[english]{babel} % The document is in English  
\usepackage[utf8]{inputenc} % UTF8 encoding
\usepackage[T1]{fontenc} % Font encoding
\usepackage[11pt]{moresize} % Big fonts

% PACKAGES FOR IMAGES
\usepackage{graphicx}
\usepackage{transparent} % Enables transparent images
\usepackage{eso-pic} % For the background picture on the title page
\usepackage{subfig} % Numbered and caption subfigures using \subfloat.
\usepackage{tikz} % A package for high-quality hand-made figures.
\usetikzlibrary{}
\graphicspath{{./Images/}} % Directory of the images
\usepackage{caption} % Coloured captions
\usepackage{xcolor} % Coloured captions
\usepackage{amsthm,thmtools,xcolor} % Coloured "Theorem"
\usepackage{float}

% STANDARD MATH PACKAGES
\usepackage{amsmath}
\usepackage{amsthm}
\usepackage{amssymb}
\usepackage{amsfonts}
\usepackage{bm}
\usepackage[overload]{empheq} % For braced-style systems of equations.
\usepackage{fix-cm} % To override original LaTeX restrictions on sizes

% PACKAGES FOR TABLES
\usepackage{tabularx}
\usepackage{longtable} % Tables that can span several pages
\usepackage{colortbl}

% PACKAGES FOR ALGORITHMS (PSEUDO-CODE)
\usepackage{algorithm}
\usepackage{algorithmic}

% PACKAGES FOR REFERENCES & BIBLIOGRAPHY
\usepackage[colorlinks=true,linkcolor=black,anchorcolor=black,citecolor=black,filecolor=black,menucolor=black,runcolor=black,urlcolor=black]{hyperref} % Adds clickable links at references
\usepackage{cleveref}
\usepackage[square, numbers, sort&compress]{natbib} % Square brackets, citing references with numbers, citations sorted by appearance in the text and compressed
\bibliographystyle{abbrvnat} % You may use a different style adapted to your field

% OTHER PACKAGES
\usepackage{pdfpages} % To include a pdf file
\usepackage{afterpage}
\usepackage{lipsum} % DUMMY PACKAGE
\usepackage{fancyhdr} % For the headers
\fancyhf{}

% Input of configuration file. Do not change config.tex file unless you really know what you are doing. 
% Define blue color typical of polimi
\definecolor{bluepoli}{cmyk}{0.4,0.1,0,0.4}

% Custom theorem environments
\declaretheoremstyle[
  headfont=\color{bluepoli}\normalfont\bfseries,
  bodyfont=\color{black}\normalfont\itshape,
]{colored}

% Set-up caption colors
\captionsetup[figure]{labelfont={color=bluepoli}} % Set colour of the captions
\captionsetup[table]{labelfont={color=bluepoli}} % Set colour of the captions
\captionsetup[algorithm]{labelfont={color=bluepoli}} % Set colour of the captions

\theoremstyle{colored}
\newtheorem{theorem}{Theorem}[chapter]
\newtheorem{proposition}{Proposition}[chapter]

% Enhances the features of the standard "table" and "tabular" environments.
\newcommand\T{\rule{0pt}{2.6ex}}
\newcommand\B{\rule[-1.2ex]{0pt}{0pt}}

% Pseudo-code algorithm descriptions.
\newcounter{algsubstate}
\renewcommand{\thealgsubstate}{\alph{algsubstate}}
\newenvironment{algsubstates}
  {\setcounter{algsubstate}{0}%
   \renewcommand{\STATE}{%
     \stepcounter{algsubstate}%
     \Statex {\small\thealgsubstate:}\space}}
  {}

% New font size
\newcommand\numfontsize{\@setfontsize\Huge{200}{60}}

% Title format: chapter
\titleformat{\chapter}[hang]{
\fontsize{50}{20}\selectfont\bfseries\filright}{\textcolor{bluepoli} \thechapter\hsp\hspace{2mm}\textcolor{bluepoli}{|   }\hsp}{0pt}{\huge\bfseries \textcolor{bluepoli}
}

% Title format: section
\titleformat{\section}
{\color{bluepoli}\normalfont\Large\bfseries}
{\color{bluepoli}\thesection.}{1em}{}

% Title format: subsection
\titleformat{\subsection}
{\color{bluepoli}\normalfont\large\bfseries}
{\color{bluepoli}\thesubsection.}{1em}{}

% Title format: subsubsection
\titleformat{\subsubsection}
{\color{bluepoli}\normalfont\large\bfseries}
{\color{bluepoli}\thesubsubsection.}{1em}{}

% Shortening for setting no horizontal-spacing
\newcommand{\hsp}{\hspace{0pt}}

\makeatletter
% Renewcommand: cleardoublepage including the background pic
\renewcommand*\cleardoublepage{%
  \clearpage\if@twoside\ifodd\c@page\else
  \null
  \AddToShipoutPicture*{\BackgroundPic}
  \thispagestyle{empty}%
  \newpage
  \if@twocolumn\hbox{}\newpage\fi\fi\fi}
\makeatother

%For correctly numbering algorithms
\numberwithin{algorithm}{chapter}

%----------------------------------------------------------------------------
%	NEW COMMANDS DEFINED
%----------------------------------------------------------------------------

% EXAMPLES OF NEW COMMANDS
\newcommand{\bea}{\begin{eqnarray}} % Shortcut for equation arrays
\newcommand{\eea}{\end{eqnarray}}
\newcommand{\e}[1]{\times 10^{#1}}  % Powers of 10 notation

%----------------------------------------------------------------------------
%	ADD YOUR PACKAGES (be careful of package interaction)
%----------------------------------------------------------------------------

%----------------------------------------------------------------------------
%	ADD YOUR DEFINITIONS AND COMMANDS (be careful of existing commands)
%----------------------------------------------------------------------------

%----------------------------------------------------------------------------
%	BEGIN OF YOUR DOCUMENT
%----------------------------------------------------------------------------

\begin{document}

\fancypagestyle{plain}{%
\fancyhf{} % Clear all header and footer fields
\fancyhead[RO,RE]{\thepage} %RO=right odd, RE=right even
\renewcommand{\headrulewidth}{0pt}
\renewcommand{\footrulewidth}{0pt}}

%----------------------------------------------------------------------------
%	TITLE PAGE
%----------------------------------------------------------------------------

\pagestyle{empty} % No page numbers
\frontmatter % Use roman page numbering style (i, ii, iii, iv...) for the preamble pages

\puttitle{
	title=WIP Title: Dynamic resource allocation in the cloud for compute heavy tasks in a containerized environment,
	name=Elia Ravella,
	course=Computer Science and Engineering,
	ID  = 967243,
	advisor= Prof. Raffaela Mirandola,
	coadvisor={Name Surname, Name Surname},
	academicyear={2021-22},
} % These info will be put into your Title page 

%----------------------------------------------------------------------------
%	PREAMBLE PAGES: ABSTRACT (inglese e italiano), EXECUTIVE SUMMARY
%----------------------------------------------------------------------------
\startpreamble
\setcounter{page}{1} % Set page counter to 1

% ABSTRACT IN ENGLISH
\chapter*{Abstract} 
\textbf{Keywords:} Cloud, Containers, Dynamic infrastructure

%----------------------------------------------------------------------------
%	LIST OF CONTENTS/FIGURES/TABLES/SYMBOLS
%----------------------------------------------------------------------------

% TABLE OF CONTENTS
\thispagestyle{empty}
\tableofcontents % Table of contents 

\addtocontents{toc}{\vspace{2em}} % Add a gap in the Contents, for aesthetics
\mainmatter % Begin numeric (1,2,3...) page numbering
	\chapter{The Problem, the State of the Art and Current Available Solutions}
		\section{Introduction}
		\label{se:intoduction}

		\section{Containerized Environment and High Performance Computing}
		\label{se:containerHPC}
    %% Section: Container and HPC
\paragraph{Containerization}
  Virtualization is "the act of creating a virtual (rather than actual) version of something at the same abstraction level, including virtual computer hardware platforms, storage devices, and computer network resources"\cite{wiki:virtualization}. In cloud environments virtualization is the most used tool to provide isolated services, as compute capabilities, storage capabilities or networking. There are different kinds of virtualization, depending on the layer virtualized: some solutions just virtualize the hardware and let the user install a full fledged OS over it, other virtualizes all the technological stack from the hardware to the OS level, leaving to the user (in most of the cases, a developer) only the problem of developing the application s/he wants to ship, without the need to care about hardware limitations or operative systems settings.\\
  
  One of the most used virtualization technique in the industry is containerization: when a software is said to be \textit{containerized} it is packaged in a format that encapsulates also all its dependencies, configuration files and variables and OS settings. This package, the container, must be run through a container engine (or equivalent operative system module) which provide the communication with the underlying software and OS to execute the software. Famous container engines are Docker, Podman, chroot or rkt, but containers were originally supported by Linux OS in form of "special processes" (the so called LXC containers). Containerized applications have several benefits over classic deployed ones:
  \begin{itemize}
    \item They do not rely on the machine or OS they are running, just the engine
    \item If the container has been built correctly, the configuration of the application is already done
    \item Packaging all their dependencies makes them independent from the other application installed on the system: two containerized applications running on the same engine could be using the same library but at different (and even incompatible) versions
    \item The container acts also as an isolation mechanism, that keeps the application from interacting with other systems: this improves the security of the container applications
  \end{itemize}
  
  If an application is containerized, moreover, it is very easy to deploy several instances of the same application working at the same time: this kind of deployments improve scalability and reliability of the application.\\

\paragraph{High Performance Computing}
  With "high performance computing" is usually addressed the field in computer science that studies computational heavy problems and develops solutions to solve them via techniques as extreme parallelism, clusterization and high performance networking. In the early stages of this project we focused also on HPC in order to understand if a virtualization approach as the containers one is suitable for such kind of tasks. The problem was the additional layer of virtualization added by the container engine: since most of the times HPC software relies heavily on low-level procedure calls, the additional virtualized layer could degrade too much the performance. However, as stated in the paper "Exploring the support for high performance applications in the container runtime environment"\cite{containershpc2} (which original aim was to compare native performance with containerized ones) the optimization of engines is reaching a level that can offer near-native performance for HPC-like loads, at least in not-extreme scenarios.

		
		\section{State of the Art}
		\label{se:stateoftheart}
    There are several solutions available on the market that provides flexible infrastructure management and are built to automate the management of the infrastructure of an application. These systems can be seen as schedulers (so software components that organize when a task is executed and on which resources) with some additional features as the capability of actually \textit{allocate} the resources needed or the automated management of the interfaces between resources and components.

\subsection{Shifter}
\label{sse:shifter}
  Shifter is a simple scheduler which aims to utilize the container format in an HPC environment. It allows the user to specify the load in a docker image, then Shifter automates the conversion of that image to an HPC format and the scheduling of such task. Shifter is \textit{not} an extension of Docker or the Docker engine, nor aims to automate the infrastructure, instead it just provides an additional interface (which is container compatible) to an already existing HPC platform.

\subsection{SLURM}
\label{sse:slurm}

\subsection{Kubernetes}
\label{sse:kubernetes}

\subsection{Serverless Approach}
\label{sse:serverless}


					
		\section{The Problem}
		\label{se:problem}
			This section highlights the problems of the currently available solutions: the focusing on scaling through replication rather than on resources size, and the problem of having a dynamical \textit{in two senses}, both resource- and replication-wise, computational layer
	
	\chapter{Design and Testing Phase}
	\label{ch:designandtesting}

		\section{MapNCloud Original Architecture}
		\label{se:originalarchitecture}
			Here I talk about the original deployment of the MapNCloud service. I plan to add a subsection explaining in detail the tech stack.

		\section{Problems Addressed}
		\label{se:problemaddressed}
			\begin{enumerate}
				\item database choice and API modification
				\item queue monitoring
				\item resizable backend containers
				\item cloud provider integration
			\end{enumerate}
			At the end of this section I will present the "final" design draft
		
		\section{Testing and Validation}
		\label{se:testingvalidation}
			\textbf{HERE} I will introduce the "diffusion analysis" to justify the test parameters
			\begin{enumerate}
				\item CouchDB testing
				\item RabbitMQ testing
				\item Cloud providers options, pros and cons
				\item technological limitations (docker-compose, load balancers)
			\end{enumerate}
			I will also present the real "final" Architecture that will be deployed here, with cloup provider's technological names and services
	
	\chapter{Implementation}
	\label{ch:implementation}

		\section{Frontend}
		\label{se:frontend}

		\section{Backend}
		\label{se:backend}

		\section{Database}
		\label{se:database}

		\section{Messaging Middleware}
		\label{se:mom}

		\section{Computational Layer}
		\label{se:complayer}
			
			\subsection{Renderino}
			\label{sse:renderino}

	\addtocontents{toc}{\vspace{2em}} % Add a gap in the Contents, for aesthetics
	\bibliography{Thesis_bibliography} % The references information are stored in the file named "Thesis_bibliography.bib"
\end{document}
