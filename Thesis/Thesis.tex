% A LaTeX template for MSc Thesis submissions to 
% Politecnico di Milano (PoliMi) - School of Industrial and Information Engineering
%
% S. Bonetti, A. Gruttadauria, G. Mescolini, A. Zingaro
% e-mail: template-tesi-ingind@polimi.it
%
% Last Revision: October 2021
%
% Copyright 2021 Politecnico di Milano, Italy. NC-BY

\documentclass{Configuration_Files/PoliMi3i_thesis}

%------------------------------------------------------------------------------
%	REQUIRED PACKAGES AND  CONFIGURATIONS
%------------------------------------------------------------------------------

% CONFIGURATIONS
\usepackage{parskip} % For paragraph layout
\usepackage{setspace} % For using single or double spacing
\usepackage{emptypage} % To insert empty pages
\usepackage{multicol} % To write in multiple columns (executive summary)
\setlength\columnsep{15pt} % Column separation in executive summary
\setlength\parindent{0pt} % Indentation
\raggedbottom  

% PACKAGES FOR TITLES
\usepackage{titlesec}
% \titlespacing{\section}{left spacing}{before spacing}{after spacing}
\titlespacing{\section}{0pt}{3.3ex}{2ex}
\titlespacing{\subsection}{0pt}{3.3ex}{1.65ex}
\titlespacing{\subsubsection}{0pt}{3.3ex}{1ex}
\usepackage{color}

% PACKAGES FOR LANGUAGE AND FONT
\usepackage[english]{babel} % The document is in English  
\usepackage[utf8]{inputenc} % UTF8 encoding
\usepackage[T1]{fontenc} % Font encoding
\usepackage[11pt]{moresize} % Big fonts

% PACKAGES FOR IMAGES
\usepackage{graphicx}
\usepackage{transparent} % Enables transparent images
\usepackage{eso-pic} % For the background picture on the title page
\usepackage{subfig} % Numbered and caption subfigures using \subfloat.
\usepackage{tikz} % A package for high-quality hand-made figures.
\usetikzlibrary{}
\graphicspath{{./Images/}} % Directory of the images
\usepackage{caption} % Coloured captions
\usepackage{xcolor} % Coloured captions
\usepackage{amsthm,thmtools,xcolor} % Coloured "Theorem"
\usepackage{float}

% STANDARD MATH PACKAGES
\usepackage{amsmath}
\usepackage{amsthm}
\usepackage{amssymb}
\usepackage{amsfonts}
\usepackage{bm}
\usepackage[overload]{empheq} % For braced-style systems of equations.
\usepackage{fix-cm} % To override original LaTeX restrictions on sizes

% PACKAGES FOR TABLES
\usepackage{tabularx}
\usepackage{longtable} % Tables that can span several pages
\usepackage{colortbl}

% PACKAGES FOR ALGORITHMS (PSEUDO-CODE)
\usepackage{algorithm}
\usepackage{algorithmic}

% PACKAGES FOR REFERENCES & BIBLIOGRAPHY
\usepackage[colorlinks=true,linkcolor=black,anchorcolor=black,citecolor=black,filecolor=black,menucolor=black,runcolor=black,urlcolor=black]{hyperref} % Adds clickable links at references
\usepackage{cleveref}
\usepackage[square, numbers, sort&compress]{natbib} % Square brackets, citing references with numbers, citations sorted by appearance in the text and compressed
\bibliographystyle{abbrvnat} % You may use a different style adapted to your field

% OTHER PACKAGES
\usepackage{pdfpages} % To include a pdf file
\usepackage{afterpage}
\usepackage{lipsum} % DUMMY PACKAGE
\usepackage{fancyhdr} % For the headers
\fancyhf{}

% Input of configuration file. Do not change config.tex file unless you really know what you are doing. 
% Define blue color typical of polimi
\definecolor{bluepoli}{cmyk}{0.4,0.1,0,0.4}

% Custom theorem environments
\declaretheoremstyle[
  headfont=\color{bluepoli}\normalfont\bfseries,
  bodyfont=\color{black}\normalfont\itshape,
]{colored}

% Set-up caption colors
\captionsetup[figure]{labelfont={color=bluepoli}} % Set colour of the captions
\captionsetup[table]{labelfont={color=bluepoli}} % Set colour of the captions
\captionsetup[algorithm]{labelfont={color=bluepoli}} % Set colour of the captions

\theoremstyle{colored}
\newtheorem{theorem}{Theorem}[chapter]
\newtheorem{proposition}{Proposition}[chapter]

% Enhances the features of the standard "table" and "tabular" environments.
\newcommand\T{\rule{0pt}{2.6ex}}
\newcommand\B{\rule[-1.2ex]{0pt}{0pt}}

% Pseudo-code algorithm descriptions.
\newcounter{algsubstate}
\renewcommand{\thealgsubstate}{\alph{algsubstate}}
\newenvironment{algsubstates}
  {\setcounter{algsubstate}{0}%
   \renewcommand{\STATE}{%
     \stepcounter{algsubstate}%
     \Statex {\small\thealgsubstate:}\space}}
  {}

% New font size
\newcommand\numfontsize{\@setfontsize\Huge{200}{60}}

% Title format: chapter
\titleformat{\chapter}[hang]{
\fontsize{50}{20}\selectfont\bfseries\filright}{\textcolor{bluepoli} \thechapter\hsp\hspace{2mm}\textcolor{bluepoli}{|   }\hsp}{0pt}{\huge\bfseries \textcolor{bluepoli}
}

% Title format: section
\titleformat{\section}
{\color{bluepoli}\normalfont\Large\bfseries}
{\color{bluepoli}\thesection.}{1em}{}

% Title format: subsection
\titleformat{\subsection}
{\color{bluepoli}\normalfont\large\bfseries}
{\color{bluepoli}\thesubsection.}{1em}{}

% Title format: subsubsection
\titleformat{\subsubsection}
{\color{bluepoli}\normalfont\large\bfseries}
{\color{bluepoli}\thesubsubsection.}{1em}{}

% Shortening for setting no horizontal-spacing
\newcommand{\hsp}{\hspace{0pt}}

\makeatletter
% Renewcommand: cleardoublepage including the background pic
\renewcommand*\cleardoublepage{%
  \clearpage\if@twoside\ifodd\c@page\else
  \null
  \AddToShipoutPicture*{\BackgroundPic}
  \thispagestyle{empty}%
  \newpage
  \if@twocolumn\hbox{}\newpage\fi\fi\fi}
\makeatother

%For correctly numbering algorithms
\numberwithin{algorithm}{chapter}

%----------------------------------------------------------------------------
%	NEW COMMANDS DEFINED
%----------------------------------------------------------------------------

% EXAMPLES OF NEW COMMANDS
\newcommand{\bea}{\begin{eqnarray}} % Shortcut for equation arrays
\newcommand{\eea}{\end{eqnarray}}
\newcommand{\e}[1]{\times 10^{#1}}  % Powers of 10 notation

%----------------------------------------------------------------------------
%	ADD YOUR PACKAGES (be careful of package interaction)
%----------------------------------------------------------------------------

%----------------------------------------------------------------------------
%	ADD YOUR DEFINITIONS AND COMMANDS (be careful of existing commands)
%----------------------------------------------------------------------------

%----------------------------------------------------------------------------
%	BEGIN OF YOUR DOCUMENT
%----------------------------------------------------------------------------

\begin{document}

\fancypagestyle{plain}{%
\fancyhf{} % Clear all header and footer fields
\fancyhead[RO,RE]{\thepage} %RO=right odd, RE=right even
\renewcommand{\headrulewidth}{0pt}
\renewcommand{\footrulewidth}{0pt}}

%----------------------------------------------------------------------------
%	TITLE PAGE
%----------------------------------------------------------------------------

\pagestyle{empty} % No page numbers
\frontmatter % Use roman page numbering style (i, ii, iii, iv...) for the preamble pages

\puttitle{
	title=WIP Title: Dynamic resource allocation in the cloud for compute heavy tasks in a containerized environment,
	name=Elia Ravella,
	course=Computer Science and Engineering,
	ID  = 967243,
	advisor= Prof. Raffaela Mirandola,
	coadvisor={Name Surname, Name Surname},
	academicyear={2021-22},
} % These info will be put into your Title page 

%----------------------------------------------------------------------------
%	PREAMBLE PAGES: ABSTRACT (inglese e italiano), EXECUTIVE SUMMARY
%----------------------------------------------------------------------------
\startpreamble
\setcounter{page}{1} % Set page counter to 1

% ABSTRACT IN ENGLISH
\chapter*{Abstract} 
\textbf{Keywords:} Cloud, Containers, Dynamic infrastructure

%----------------------------------------------------------------------------
%	LIST OF CONTENTS/FIGURES/TABLES/SYMBOLS
%----------------------------------------------------------------------------

% TABLE OF CONTENTS
\thispagestyle{empty}
\tableofcontents % Table of contents 

\addtocontents{toc}{\vspace{2em}} % Add a gap in the Contents, for aesthetics
\mainmatter % Begin numeric (1,2,3...) page numbering
	\chapter{The Problem, the State of the Art and Current Available Solutions}
		\section{Introduction}
		\label{se:introduction}
    
    \section{The Problem}
		\label{se:problem}
    Scalability is one of the key design point that must be taken into consideration when developing a software. If a system cannot scale in power when the userbase or the load requested changes it slows down, making the response times growing for each request and compromising the overall performance of the application. The most intuitive approach to scalability, which is also the most common in cloud environments, is horizontal replication. With horizontal replication I mean (throughout all this thesis) the addition of identical software modules alongside the already existing ones to share the load; to do so, different incoming requests are routed to different modules when they arrive. The replicas being identical (and usually stateless) ensure that each request is carried out in the same way. The policies for deploying replicas can be either static or dynamic (based on the predicted load during the day or measuring the real time traffic incoming, for example).\\

This approach has seen a wide adoption in the industry and is the \textit{de facto} standard to tackle scalability problems, especially in web environments. The horizontal replication approach gives applications the flexibility they need in reacting to the load that is applied, and is especially effective when the application is divided in submodules that can be individually scaled.\\

What happens when the requests that an application must serve change also in nature, and not only in volume? Horizontal replication works well when the load is mostly uniform (and for the majority of web applications, it is) and can be analyzed in a one dimensional fashion as "the number of requests". When requests set in motion heavy computational pipelines, as image processing or complex mathematical problems, but the interface they are served is shared with all the other \textit{light} requests\footnote{this is not an impossible scenario: what a REST API exposes are a list of "light" HTTP requests that can trigger all kind of operations on the server they are executed on} then a single request can weight, in terms of resources it needs to be carried out, very differently from the others. In this scenario, horizontal replication is harder to put in place effectively: if requests weight differently it is not possible to just share them equally among replicated servers\footnote{here \textit{servers} is used to describe logical backend modules, not physical machines} because in some cases a server will receive a much more higher share of heavy requests and be stuck executing them while other servers will be idling because they received only light requests. Horizontal replication, as it is implemented now, cannot face efficently this scenario. The proposed solution uses dedicated ephemeral workers to execute \textit{heavy} requests, and organises and schedules them with a ticketing middleware. 

	
		\section{Containerized Environment and High Performance Computing}
		\label{se:containerHPC}
    %% Section: Container and HPC
\paragraph{Containerization}
  Virtualization is "the act of creating a virtual (rather than actual) version of something at the same abstraction level, including virtual computer hardware platforms, storage devices, and computer network resources"\cite{wiki:virtualization}. In cloud environments virtualization is the most used tool to provide isolated services, as compute capabilities, storage capabilities or networking. There are different kinds of virtualization, depending on the layer virtualized: some solutions just virtualize the hardware and let the user install a full fledged OS over it, other virtualizes all the technological stack from the hardware to the OS level, leaving to the user (in most of the cases, a developer) only the problem of developing the application s/he wants to ship, without the need to care about hardware limitations or operative systems settings.\\
  
  One of the most used virtualization technique in the industry is containerization: when a software is said to be \textit{containerized} it is packaged in a format that encapsulates also all its dependencies, configuration files and variables and OS settings. This package, the container, must be run through a container engine (or equivalent operative system module) which provide the communication with the underlying software and OS to execute the software. Famous container engines are Docker, Podman, chroot or rkt, but containers were originally supported by Linux OS in form of "special processes" (the so called LXC containers). Containerized applications have several benefits over classic deployed ones:
  \begin{itemize}
    \item They do not rely on the machine or OS they are running, just the engine
    \item If the container has been built correctly, the configuration of the application is already done
    \item Packaging all their dependencies makes them independent from the other application installed on the system: two containerized applications running on the same engine could be using the same library but at different (and even incompatible) versions
    \item The container acts also as an isolation mechanism, that keeps the application from interacting with other systems: this improves the security of the container applications
  \end{itemize}
  
  If an application is containerized, moreover, it is very easy to deploy several instances of the same application working at the same time: this kind of deployments improve scalability and reliability of the application.\\

\paragraph{High Performance Computing}
  With "high performance computing" is usually addressed the field in computer science that studies computational heavy problems and develops solutions to solve them via techniques as extreme parallelism, clusterization and high performance networking. In the early stages of this project we focused also on HPC in order to understand if a virtualization approach as the containers one is suitable for such kind of tasks. The problem was the additional layer of virtualization added by the container engine: since most of the times HPC software relies heavily on low-level procedure calls, the additional virtualized layer could degrade too much the performance. However, as stated in the paper "Exploring the support for high performance applications in the container runtime environment"\cite{containershpc2} (which original aim was to compare native performance with containerized ones) the optimization of engines is reaching a level that can offer near-native performance for HPC-like loads, at least in not-extreme scenarios.

		
		\section{State of the Art}
		\label{se:stateoftheart}
    There are several solutions available on the market that provides flexible infrastructure management and are built to automate the management of the infrastructure of an application. These systems can be seen as schedulers (so software components that organize when a task is executed and on which resources) with some additional features as the capability of actually \textit{allocate} the resources needed or the automated management of the interfaces between resources and components.

\subsection{SLURM}
\label{sse:slurm}
  SLURM is "an open source, fault-tolerant, and highly scalable cluster management and job scheduling system for large and small Linux clusters"\cite{slurm}. It aims to organize and schedule tasks on multiple nodes; these tasks can also be defined as OCI-compliant containers. SLURM was crated to be executed on supercomputers or clusters of computers; in fact, SLURM focuses on communication between daemons and tasks through message passing framework as MPI and queue-managed resource access. Even if it can integrate containerized workloads, it is not suited to be deployed in a cloud environment rather than a computation center.

\subsection{Shifter}
\label{sse:shifter}
  Shifter is a simple scheduler which aims to utilize the container format in an HPC environment. It allows the user to specify the load in a docker image, then Shifter automates the conversion of that image to an HPC format and the scheduling of such task. Shifter is \textit{not} an extension of Docker or the Docker engine, nor aims to automate the infrastructure, instead it just provides an additional interface (which is container compatible) to an already existing HPC platform.

\subsection{Kubernetes}
\label{sse:kubernetes}
  Kubernetes (often called "k8s") is a container orchestration system, and is the \textit{de facto} standard for container orchestration. It provides an all-in-one system to manage containerized applications:
  \begin{itemize}
    \item It provides an abstraction over the container level (the Pod) that is used to define the service provided rather than the container itself
    \item It includes different ways to persist data and state across containers; this gives a kubernetes cluster the capability to hold an entire application, from data layer to presentation layer
    \item Kubernetes clusters embeds security and access control by enforcing the already existing isolation features of a containerized environment and by providing a set of tools that easily control the access to the cluster itself (the Ingress controllers)
    \item It allows developers to define the redundancy for every single service defined, so to set an "horizontal scaling width" beforehand to handle faults and heavy loads
  \end{itemize}
  Kubernetes has been designed for applications that are designed as microservices, and it provides the tooling for administrate single services, replication, and scaling in such an environment. As already stated in (\ref{se:problem}) the level of abstraction provided by kubernetes is efficent when horizontal scaling must be automated, but also removes some of the controllability of a containerized system. Moreover, resource management within kubernetes deployment in the cloud is very difficult: all the needed resources must be available at any time so that kubernetes can manage them, and this implies reserving significant amounts of resources for a prolonged period of time; in a cloud deployed application this renders the cost of application maintenance too high. 

\subsection{Serverless Approach}
\label{sse:serverless}

			
		
	\chapter{Design and Testing Phase}
	\label{ch:designandtesting}

		\section{MapNCloud Original Architecture}
		\label{se:originalarchitecture}
			Here I talk about the original deployment of the MapNCloud service. I plan to add a subsection explaining in detail the tech stack.

		\section{Problems Addressed}
		\label{se:problemaddressed}
			\begin{enumerate}
				\item database choice and API modification
				\item queue monitoring
				\item resizable backend containers
				\item cloud provider integration
			\end{enumerate}
			At the end of this section I will present the "final" design draft
		
		\section{Testing and Validation}
		\label{se:testingvalidation}
			\textbf{HERE} I will introduce the "diffusion analysis" to justify the test parameters
			\begin{enumerate}
				\item CouchDB testing
				\item RabbitMQ testing
				\item Cloud providers options, pros and cons
				\item technological limitations (docker-compose, load balancers)
			\end{enumerate}
			I will also present the real "final" Architecture that will be deployed here, with cloup provider's technological names and services
	
	\chapter{Implementation}
	\label{ch:implementation}

		\section{Frontend}
		\label{se:frontend}

		\section{Backend}
		\label{se:backend}

		\section{Database}
		\label{se:database}

		\section{Messaging Middleware}
		\label{se:mom}

		\section{Computational Layer}
		\label{se:complayer}
			
			\subsection{Renderino}
			\label{sse:renderino}

	\addtocontents{toc}{\vspace{2em}} % Add a gap in the Contents, for aesthetics
	\bibliography{Thesis_bibliography} % The references information are stored in the file named "Thesis_bibliography.bib"
\end{document}
