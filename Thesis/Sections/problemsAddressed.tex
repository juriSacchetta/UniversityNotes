The original application was packaged and deployed as a self-hostable stack: a group of containerized applications all in the same configuration files, which used docker-compose as the orchestration tool. This simple deployment solution was suited for the demo required for the original project, but the really limited support cloud providers offer to the docker-compose tool\footnote{at the moment of writing, several cloud providers has early-development stage support for docker compose, moreover further limited to only a few container-oriented services.} and the future necessity to dynamically scale the backend modules rendered this approach non suitable. In this section we will address the major problems the architecture had in the first place, how these were faced and which decisions were taken to overcome them.

\subsection{Scalability}
  The MapNCloud system was designed to ensure a detachment between the "business logic" backend and the "computational" backend, in order to ensure that that the full application does not get stuck on a request by offloading longer tasks to a different independent module. This module (\ref{ssse:originalcomputationlayer}) is the main bottleneck of the system, and though the one I focused on to improve its overall scalability.\\
  The first approach considered was to use an automated orchestration software as Kubernetes (\ref{sse:kubernetes}) or a task scheduler as SLURM (\ref{sse:slurm}) to manage the mapping of tasks on available computational units; as explained in the respective sections, these approaches were not viable, due to either an high infrastructure cost or a poor cloud compatibility. The system was required to be able to handle different traffic paces, with different requests weights, being at the same time cost effective\footnote{which translates, in cloud environments, to deallocate unused resources}.\\
  The target solution needed to dynamically spawn workers, each one tailored to the actual traffic detected or to a forecast on traffic (based on past data or statically, using time of day as index for example). Each worker had to be spawned with a different set of resources, to better handle the traffic. As an example, if a great number of requests suddenly arrive all together, each one with an high number of input files, some resourceful workers needed to be spawned; instead, if the number of unacknowledged tasks grows steadily but slowly, a smaller\footnote{smaller in the number or size of resources allocated: number of CPUs, amount of RAM and optional GPUs} worker unit can be created to handle the accumulating tasks. This kind of reasoning \textit{must} be applied also in the other direction: especially in the cloud, systems that allocates a significant number of hardware resources are very expensive; if heavy workers (i.e. some modules with high CPU or GPU count) are idling, or the system can satisfy the incoming traffic without them, in order to be cost effective the system should be able to identify the unnecessary workers and kill them. The final design uses an automation tool provided by the cloud vendor to expose a REST interface that allows an alerting module (built in the queuing system) to notify congestion situations, and to also shape the worker spawned in terms of CPU count, GPU count and memory available.

\subsection{Cloud Provider Integration}
  The choice of the cloud provider was discussed in the first steps of the project, and after some initial performance tests Azure by Microsoft has been chosen. These simple tests were conducted to see if there was a significant difference in performance between the two vendors: the results are available in the dedicated appendix. In the choice of the cloud vendor other factors have been relevant, as the ease of integration between internal services, general support provided by the vendor itself, migration problems and developer familiarity with the tools.

\subsection{Queue Monitoring}
  The queue module is the one that is "congestion aware". The workers are designed to interact with it by dequeuing a message (which contains all the task-related information), processing the task and only when the task has been successfully executed and completed acknowledge the message to the queue. This provides a straightforward metric to both understand congestion of the queues and of the workers: the combination of unacknowledged messages and the ones entirely not delivered to clients. A majority of unacknowledged messages would suggest that the workers are underpowered, while a majority of not delivered messages would suggest that too few workers are deployed. These kind of metrics and indexes must be extracted from the queuing system in order to manage the number of workers present.\\
  Luckily, RabbitMQ embeds a monitoring tool (called Prometheus) which automatically organizes the queues data and exports them in a text format. An additional module of Prometheus, called AlertManager, handles the notifications and interacts with external services; this fits perfectly our use case, allowing us to group together various pieces of information and deliver a precise alert (which will be encapsulated in a HTTP request) that actively reshapes the backend to align to the congestion status.\\
  In the early design steps, it was considered to use a cluster of RabbitMQ instances rather than a single one, to improve on reliability and resilience. The cluster additional copies of RabbitMQ would have provided an higher failure tolerance, since the whole cluster could then withstand a failure of an instance, up to "all but one" instances failure. This option was discarded:
  \begin{itemize}
    \item Using a set of instances in place of a single one requires either to inform the other modules of the fallback options or to use a load balancer put in front of the RabbitMQ cluster to interact with it: in the former case all the backend modules would have needed an additional modification; in the latter, an additional software component should have been inserted in the architecture and configured, with the risk of degrading performance
    \item Tests were conducted to assure if a single RabbitMQ instance could withstand the load imposed by the application, in different scenarios (as single backend workers versus dynamic resizing backend workers). These tests are available in the appendix
    \item Clustering RabbitMQ nodes, although a very powerful feature of that software and surely fundamental in large distributed message oriented scenarios, requires some additional per-node configuration which would have itself complicated the deployment of the system
  \end{itemize}

\subsection{API Redesign and Organization}
  The module that mostly is impacted by this thesis' work is the backend. From the original design (\ref{ssse:originalcomputationlayer}) the module has been separated in two main components
  \begin{enumerate}
    \item the \textbf{MainBackend}: this module actually hosts the exposed Application Programming Interface the frontend utilizes. The interface exposed is very similar to the original one, the major changes residing in the data access layer\footnote{further details on the original architecture available in appendix (\ref{ap:originaldesigndoc})} which no longer interfaces directly with the database, but instead calls another API to interact with it for tasks as attachments searching and retrieving, data access (which encapsulates authentication) and uploading of resources.\\
      This piece is software is deployed as an Azure Web App exposed on the internet (which is in contrast with most of the other components of the architecture, which are deployed inside a virtual network) and is packaged as a container. This is the module that encapsulates the business logic of the application. It exposes an interface to the frontend to be utilized and interacts with the OperationalBackend to handle the data (images, models, user data) and the task management functions, as interfacing with RabbitMQ to add a task to be executed. It interacts with the other components just through REST interfaces.
    \item the \textbf{OperationalBackend}: this module is deployed as an Azure Web App also, and moreover it is assigned at the same hosting and scaling plan of the Main one (which are called Azure App Services); differently from the Main one, though, it is deployed inside a dedicated subnet of the Virtual Network used to isolate the system. This module however is additionally secured, allowing among all incoming requests only the one originating from the MainBackend to actually be processed. This has two major benefits: firstly, making sure that only requests from a "secure" origin can penetrate the virtual network and reach the OperationalBackend, and then to avoid the waste of resources to process "useless" requests.\\
      Like the Main one, this backend is packaged and deployed as a Docker container; being this the component that also initializes the database and the queue system, and so uses many different configuration and initialization files, the docker packaging mechanism is useful to keep all the app configuration organized together.\\
      This component has a variety of roles:
      \begin{itemize}
        \item it exposes a REST interface for the MainBackend that implements the basic CRUD\footnote{Create Read Update Delete} operations for the data models of the applications, acting like a "proxy" of the database
        \item the same API has a set of endpoints dedicated to the computational layer modules, the "renderini", to operate (as an example) group resources handling
        \item it interfaces with RabbitMQ via AMQP channels to add tasks for the computational layer to be executed
        \item it needs to connect to the CouchDB instance in order to expose its functionalities, and it does so through the dedicated driver offered by Apache itself, Nano, which abstracts the native CouchDB REST API and provides an easy to use TypeScript interface to the database data model
      \end{itemize}
  \end{enumerate}
  This separation has been done for several reasons; one of the first design choices that was made which regarded the internal architecture was to remove the renderini-database interaction, and having instead them asking for the data directly to the backend, breaking the functions loop (\ref{ssse:originalcomputationlayer}) to simplify the architecture and make it more fault tolerant. The problem that arose was that a single module, the backend, was taking care of the frontend load and the renderini load, so it would fast become a bottleneck. As per the separation of the computation layer, mentioned in (\ref{sse:originalsystemdesign}), the solution was to separate the functionalities in two different modules, in order to have two parallel components that do not interfere with each other.\\
  Another reason that led to this separation was to restrict even more the access to the virtual network: Azure allows for web apps to restrict the incoming traffic in a very fine grained manner, and in this case this feature allowed us not to restrict only the port or addresses pool but even the specific client that could interact with the API, securing even more our perimeter (which already restricts inbound and outbound traffic with Access Control List-like policies provided by Azure, called Network Security Groups rules).
