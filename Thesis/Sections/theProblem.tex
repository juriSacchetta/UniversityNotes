Scalability is one of the key design point that must be taken into consideration when developing a software. If a system cannot scale in power when the userbase or the load requested changes it slows down, making the response times growing for each request and compromising the overall performance of the application. The most intuitive approach to scalability, which is also the most common in cloud environments, is horizontal replication. With horizontal replication I mean (throughout all this thesis) the addition of identical software modules alongside the already existing ones to share the load; to do so, different incoming requests are routed to different modules when they arrive. The replicas being identical (and usually stateless) ensure that each request is carried out in the same way. The policies for deploying replicas can be either static or dynamic (basing on the predicted load during the day or measuring the real time traffic incoming, for example).\\

This approach has seen a wide adoption in the industry and is the \textit{de facto} standard to tackle scalability problems, especially in web environments. The horizontal replication approach gives applications the flexibility they need in reacting to the load that is applied, and is especially effective when the application is divided in submodules that can be individually scaled.\\

What happens when the requests that an application must serve change also in nature, and not only in volume? Horizontal replication works well when the load is mostly uniform (and for the majority of web applications, it is) and can be analyzed in a one dimensional fashion as "the number of requests". %% TODO go on alking about resizing the hardware and resources
